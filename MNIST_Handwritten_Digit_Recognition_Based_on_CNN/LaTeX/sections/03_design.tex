\section{实验设计}\label{sec:design}

\subsection{数据预处理}

在实验中,我们对MNIST数据集进行了简单的预处理。首先,我们将图像数据转换为张量,并进行了归一化处理,使像素值在[0, 1]范围内。这有助于加速模型收敛并提高模型性能。

\subsection{CNN模型训练}

我们使用了包含两个卷积层和两个全连接层的CNN模型。模型的参数初始化采用了默认的方法。在训练过程中,我们采用了批量梯度下降(Batch Gradient Descent)方法,每次从训练集中随机采样一批数据进行模型参数更新。同时,我们使用了Adam优化器来自适应地调整学习率。

\subsection{实验评估}

在训练过程中,我们通过监测训练损失的变化来评估模型的训练情况,并在每个epoch结束时计算了平均训练损失。此外,我们在测试集上评估了模型的性能,计算了模型在测试集上的准确率以及平均测试损失。我们还通过绘制损失随epoch变化的曲线和展示部分测试集样本的预测结果,来直观地展示模型的训练情况和性能表现。