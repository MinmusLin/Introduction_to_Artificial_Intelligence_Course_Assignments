\section{总结}\label{sec:conclusion}

本实验采用了CNN模型对MNIST手写数字数据集进行训练和测试。通过对模型训练过程的分析,观察到模型在训练集上表现出良好的收敛性和稳定性,随着训练的进行,平均损失逐渐减小,证明了模型在不断优化参数以适应数据的过程中的有效性。在测试集上评估模型性能,获得了99.19\%的准确率和较小的平均损失,验证了模型在未见数据上的泛化能力和预测准确性。综合考虑训练和测试结果,表明所构建的CNN模型在手写数字识别任务上表现出色,具有实际应用的潜力和稳定性,为解决图像分类问题提供了可靠的解决方案。

通过本实验,我对深度学习和卷积神经网络有了更深入的理解。深度学习模型的训练过程需要耐心和细致的调整,同时了解CNN的结构和原理能够更好地指导模型设计和调优。这次实验让我更加确信深度学习技术的潜力和广泛应用的前景,也激励我继续深入学习和探索。