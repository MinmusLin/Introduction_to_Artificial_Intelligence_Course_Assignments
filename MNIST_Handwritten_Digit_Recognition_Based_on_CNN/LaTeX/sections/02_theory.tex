\section{实验原理}\label{sec:theory}

\subsection{CNN(卷积神经网络)}

卷积神经网络(Convolutional Neural Network,CNN)是一种深度学习模型,专门用于处理具有网格结构的数据,如图像。CNN在图像识别任务中表现出色,因其能够自动学习图像中的特征而广受欢迎。下面是本实验中使用的CNN结构,共有421642个参数。

\subsubsection{卷积层(Convolutional Layer)}

卷积层是CNN的核心组件之一。它通过在图像上滑动卷积核(filter)来提取图像的特征。每个卷积核都包含一组权重,可以学习不同的特征模式。本实验中,我们定义了两个卷积层:

第一个卷积层:输入通道为1(灰度图像),输出通道为32,卷积核大小为3x3。

第二个卷积层:输入通道为32,输出通道为64,卷积核大小为3x3。

\subsubsection{池化层(Pooling Layer)}

池化层用于减少卷积层输出的空间维度,同时保留重要信息。在本实验中,我们采用最大池化(Max Pooling)操作,将每个2x2的区域中的最大值作为输出。

\subsubsection{全连接层(Fully Connected Layer)}

全连接层用于将卷积层输出的特征图转换为分类结果。在本实验中,我们定义了两个全连接层:

第一个全连接层:输入特征维度为64x7x7(64个通道,每个通道大小为7x7),输出维度为128。

第二个全连接层:输入维度为128,输出维度为10(对应10个数字类别)。

\subsection{损失函数与优化器}

在训练CNN模型时,需要定义损失函数来衡量模型输出与真实标签之间的差距,并使用优化器来更新模型参数以最小化损失函数。在本实验中,我们采用以下损失函数和优化器:

损失函数:交叉熵损失函数(Cross Entropy Loss),用于多分类问题的损失计算。

优化器:Adam优化器,一种自适应学习率的优化算法,用于更新模型参数。

\begin{listing}[htbp]
  \begin{minted}{python}
class CNNModel(nn.Module):
    def __init__(self):
        super(CNNModel, self).__init__()
        self.conv1 = nn.Conv2d(1, 32, kernel_size=3, padding=1)
        self.conv2 = nn.Conv2d(32, 64, kernel_size=3, padding=1)
        self.pool = nn.MaxPool2d(2, 2)
        self.fc1 = nn.Linear(64 * 7 * 7, 128)
        self.fc2 = nn.Linear(128, 10)
    def forward(self, x):
        x = self.pool(F.relu(self.conv1(x)))
        x = self.pool(F.relu(self.conv2(x)))
        x = x.view(-1, 64 * 7 * 7)
        x = F.relu(self.fc1(x))
        x = self.fc2(x)
        return F.log_softmax(x, dim=1)
model = CNNModel()
criterion = nn.CrossEntropyLoss()
optimizer = optim.Adam(model.parameters(), lr=0.001, weight_decay=1e-5)
  \end{minted}
  \caption{卷积神经网络模型定义代码段}
\end{listing}