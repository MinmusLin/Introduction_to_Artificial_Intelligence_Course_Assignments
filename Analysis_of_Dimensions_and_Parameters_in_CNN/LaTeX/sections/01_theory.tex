\section{分析原理}\label{sec:theory}

\subsection{卷积层(CONV k-N)}

卷积层通过在输入数据上滑动多个卷积核(或过滤器)来提取特征。每个卷积核负责学习输入数据的一种特征。

$k$表示卷积核的大小(高度和宽度),$N$表示卷积核的数量。每个卷积核有$k \times k \times C_{in}$个权重参数,其中$C_{in}$是输入通道数。加上偏置项(每个卷积核一个),总参数量为$((k \times k \times C_in)+1) \times N$。

当使用$padding=1$和$stride=1$时,输出高度和宽度与输入相同,输出通道数为卷积核数量$N$。因此,输出维度是$H \times W \times N$。

\subsection{池化层(POOL-n)}

池化层用于降低特征图的空间尺寸(高度和宽度),以减少计算量并防止过拟合。

池化层没有参数。

使用$n \times n$的窗口和$stride=n$(通常没有$padding$),输出的高度和宽度是输入的$1/n$,输出通道数不变。输出维度是$H/n \times W/n \times C$。

\subsection{全连接层(FC-N)}

全连接层将学习到的“特征”表示映射到最终的输出,如分类标签。每个输入节点都连接到输出节点。

如果输入被展平为$D$个元素,输出为$N$个节点,则参数量为$(D+1) \times N$,其中$+1$代表偏置项。

输出是一个$N$维向量,因此维度为$1 \times 1 \times N$。

\subsection{Leaky ReLU}

Leaky ReLU(Leaky Rectified Linear Unit)是ReLU激活函数的一个变种,允许小的梯度值流过,防止神经元完全死亡。其公式为
$
f(x) = \begin{cases} 
	x & \text{if } x > 0 \\
	\alpha x & \text{otherwise}
\end{cases}
$
,其中$\alpha$是一个很小的常数。

Leaky ReLU没有参数。

不改变输入的尺寸,因此输出维度与输入维度相同。

\subsection{FLATTEN}

Flatten层将多维的输入展平为一维,通常用在卷积层和全连接层之间。

Flatten层没有参数。

如果输入维度是$H \times W \times C$,输出维度为$1 \times 1 \times (H \times W \times C)$。