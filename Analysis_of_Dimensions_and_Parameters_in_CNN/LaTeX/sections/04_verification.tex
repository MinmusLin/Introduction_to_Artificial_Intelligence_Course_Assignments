\section{结果验证}\label{sec:verification}

通过\cref{code_define}定义CNN模型,并通过\cref{code_print}打印CNN模型参数量,可以得到以下输出结果,结果验证正确。

\begin{listing}[htbp]
	\begin{minted}{text}
conv1.weight has 432 parameters
conv1.bias has 16 parameters
conv2.weight has 4608 parameters
conv2.bias has 32 parameters
fc.weight has 15680 parameters
fc.bias has 10 parameters
total parameters: 20778
	\end{minted}
\end{listing}

\begin{listing}[htbp]
  \begin{minted}{python}
class MyCNN(nn.Module):
    def __init__(self):
        super(MyCNN, self).__init__()
        self.conv1 = nn.Conv2d(in_channels=3, out_channels=16, kernel_size=3, padding=1)
        self.conv2 = nn.Conv2d(in_channels=16, out_channels=32, kernel_size=3, padding=1)
        self.pool = nn.MaxPool2d(kernel_size=2, stride=2)
        self.fc = nn.Linear(in_features=7 * 7 * 32, out_features=10)

    def forward(self, x):
        x = F.leaky_relu(self.conv1(x))
        x = self.pool(x)
        x = F.leaky_relu(self.conv2(x))
        x = self.pool(x)
        x = torch.flatten(x, 1)
        x = self.fc(x)
        return x
  \end{minted}
  \caption{定义CNN模型代码段}\label{code_define}
\end{listing}

\begin{listing}[htbp]
	\begin{minted}{python}
total_params = 0
for name, parameter in model.named_parameters():
    if not parameter.requires_grad: continue
    param = parameter.numel()
    print(f'{name} has {param} parameters')
    total_params += param
print(f'total parameters: {total_params}')
	\end{minted}
	\caption{打印CNN模型参数量代码段}\label{code_print}
\end{listing}